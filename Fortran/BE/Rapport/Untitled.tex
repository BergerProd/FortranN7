\documentclass[a4paper,oneside]{article}
\usepackage[french]{babel}
\usepackage[utf8]{inputenc}
\usepackage{hyperref} % références dans pdf
\usepackage[tt]{titlepic}
\usepackage{graphicx} % pour images
\usepackage{rotating} % pour
\usepackage{lmodern}
\usepackage{amsmath}
\usepackage{amssymb}
\usepackage{mathrsfs}
\usepackage{sistyle}
\usepackage{chngpage}
\usepackage{epstopdf}
\usepackage{gnuplottex}% pour faire du gnuplot directement dans le latex, finalement pas utilisé, résultats pas assez beaux
\usepackage[nottoc, notlof, notlot]{tocbibind} % pour que bibliographie soit comprise comme un chapitre ou section
\usepackage{appendix} % pour les annexes
\pagestyle{headings} % pour en têtes

\makeatletter % pour /bigcenter qui permet de s'affranchir des marges pour les images

\newskip\@bigflushglue \@bigflushglue = -100pt plus 1fil

\def\bigcenter{\trivlist \bigcentering\item\relax}
\def\bigcentering{\let\\\@centercr\rightskip\@bigflushglue%
\leftskip\@bigflushglue
\parindent\z@\parfillskip\z@skip}
\def\endbigcenter{\endtrivlist}

\makeatother

\begin{document}

%************************************************************************
%									TITRE
%************************************************************************


\begin{titlepage} % Suppresses headers and footers on the title page

	\centering % Centre everything on the title page

	\scshape % Use small caps for all text on the title page

	\vspace*{\baselineskip} % White space at the top of the page


	\rule{\textwidth}{1.6pt}\vspace*{-\baselineskip}\vspace*{2pt}
	 % Thick horizontal rule
	\rule{\textwidth}{0.4pt} % Thin horizontal rule

	\vspace{0.75\baselineskip} % Whitespace above the title

	{\LARGE RAPPORT DE VALIDATION :\\
	BUREAU D'\'ETUDES \\
	\vspace{0.75\baselineskip}
	Balistique \& Trajectoire} % Title

	\vspace{1\baselineskip} % Whitespace below the title
	\rule{\textwidth}{0.4pt}\vspace*{-\baselineskip}\vspace*{3.2pt}
	 % Thin horizontal rule
	\rule{\textwidth}{1.6pt} % Thick horizontal rule
	\vspace{2\baselineskip} % Whitespace after the title block

	%------------------------------------------------
	%	Subtitle
	%------------------------------------------------

	% Subtitle or further description
	Calculs Scientifiques \& Programmation

	\vspace*{3\baselineskip} % Whitespace under the subtitle

	%------------------------------------------------
	%	Editor(s)
	%------------------------------------------------


	\vspace{0.5\baselineskip} % Whitespace before the editors

	{\scshape\Large Quentin Bergé \\ Marc Ferrière} % Editor list

	\vspace{0.5\baselineskip} % Whitespace below the editor list

	\textit{ENSEEIHT} % Editor affiliation

	\vfill % Whitespace between editor names and publisher logo

	%------------------------------------------------
	%	Publisher
	%------------------------------------------------

	\includegraphics[scale=0.3]{logoN7.png} % changer logo

	\vspace{0.3\baselineskip} % Whitespace under the publisher logo

Fevrier 2019 % Publication year



\end{titlepage}
\newpage

\tableofcontents
\newpage

%*******************************************************
% DEBUT RAPPORT
%********************************************************


\section{Introduction}
Il est question ici de créer un programme pour effectuer le calcul de trajectoire d'un objet en chute libre et propulsé selon l'enoncé proposé.
Dans ce rapport nous aborderons tout d'abord la modélisation du problème, puis nous nous focaliserons sur l'architecture du programme pour enfin aborder la validation.

\section{Modélisation}

\subsection{Chute Libre}

Si on effectue le bilan des forces sur un objet de masse $m$ possédant une vitesse initiale $v_0$ formant un angle $\alpha$ avec l'axe des abcisses à une altitude initiale $h$ :

\[
\sum \overrightarrow{F} = m \times \overrightarrow{a}
\]

\subsection{Objet Portant Propulsé}




\section{Architecture}

\subsection{Contenu}

Le programme est placé dans le dossier Fortran qui comprend :

\begin{itemize}
	\item un fichier \verb?mod_balistique.f90? % pour ne pas tenir compte de la mise en forme du texte très utile !!!
	\item un fichier de subroutines \verb?subroutines.f90?
	\item le programme \verb?main.f90?
	\item le fichier \verb?Makefile? permettant la compilation\\
	\end{itemize}

Le dossier RUN comprend :
\begin{itemize}
	\item les fichier de sorties de forme \verb?BE_Methode_Modelisation_npt_XXXX_.out? et \verb? parametrisation_alpha.out?
	\item le fichier d'entrée \verb?balistique.in?
	\item le programme compilé \verb?main.bin?\\
\end{itemize}

Dans le dossier Validation sont compris :
\begin{itemize}
	\item les résultats du programme pour les différents modèles avec les deux modélisation
	\item les graphiques de résultats\\
\end{itemize}

Le dossier Rapport comprend les ressources pour le rapport \LaTeX  ainsi que le rapport en lui-même.

\subsection{Algorithmie}
%algorithme du code à faire ou à mettre en annexe



\section{Validation}

On effectue la validation selon la solution analytique pour le cas de la chute libre pour les deux méthodes.

Tout d'abord selon la méthode d'Euler.

\begin{figure}[htbp]
\centering
\begin{gnuplot}[terminal=latex]
set size 1,1; set title "Validation Chute Libre Euler";  plot "BE_Euler_Chute_Libre_npt_10000.out" using 3:4 title "Méthode Euler" with lines lt rgb "violet", "BE_Euler_Chute_Libre_npt_10000.out" using 5:6 title "Méthode Analytique" with lines
\end{gnuplot}
\caption{Validation de la Chute Libre selon la méthode d'Euler}
\end{figure}

On observe la parfaite superposition entre les deux courbes, ainsi on peut valider nos résultats avec la résolution de la méthode d'Euler.


\section{Conclusion}


\end{document}
