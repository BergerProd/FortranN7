\documentclass[a4paper,oneside]{article}
%Compilé avec MacTex
\usepackage[french]{babel}
\usepackage[utf8]{inputenc}
\usepackage{hyperref} % références dans pdf
\usepackage[tt]{titlepic}
\usepackage{graphicx} % pour images
\usepackage{rotating} % pour rotatebox
\usepackage{lmodern}
\usepackage{amsmath}
\usepackage{amssymb}
\usepackage{mathrsfs}
\usepackage{sistyle}
\usepackage{chngpage}
\usepackage{epstopdf}
\usepackage{gnuplottex}% pour faire du gnuplot directement dans le latex
\usepackage[nottoc, notlof, notlot]{tocbibind} % pour que bibliographie soit comprise comme un chapitre ou section
\usepackage{appendix} % pour les annexes
\pagestyle{headings} % pour en têtes
\usepackage{mathenv}
\DeclareTextSymbol{\degre}{OT1}{23} %pour le symbole degré
\makeatletter % pour /bigcenter qui permet de s'affranchir des marges pour les images
\newskip\@bigflushglue \@bigflushglue = -100pt plus 1fil

\def\bigcenter{\trivlist \bigcentering\item\relax}
\def\bigcentering{\let\\\@centercr\rightskip\@bigflushglue%
\leftskip\@bigflushglue
\parindent\z@\parfillskip\z@skip}
\def\endbigcenter{\endtrivlist}
\makeatother


\begin{document}

%************************************************************************
%									TITRE
%************************************************************************


\begin{titlepage} % Suppresses headers and footers on the title page

	\centering % Centre everything on the title page

	\scshape % Use small caps for all text on the title page

	\vspace*{\baselineskip} % White space at the top of the page


	\rule{\textwidth}{1.6pt}\vspace*{-\baselineskip}\vspace*{2pt}
	 % Thick horizontal rule
	\rule{\textwidth}{0.4pt} % Thin horizontal rule

	\vspace{0.75\baselineskip} % Whitespace above the title

	{\LARGE RAPPORT DE VALIDATION :\\
	BUREAU D'\'ETUDES \\
	\vspace{0.75\baselineskip}
	Balistique \& Trajectoire} % Title

	\vspace{1\baselineskip} % Whitespace below the title
	\rule{\textwidth}{0.4pt}\vspace*{-\baselineskip}\vspace*{3.2pt}
	 % Thin horizontal rule
	\rule{\textwidth}{1.6pt} % Thick horizontal rule
	\vspace{2\baselineskip} % Whitespace after the title block

	%------------------------------------------------
	%	Subtitle
	%------------------------------------------------

	% Subtitle or further description
	Calculs Scientifiques \& Programmation

	\vspace*{3\baselineskip} % Whitespace under the subtitle

	%------------------------------------------------
	%	Editor(s)
	%------------------------------------------------


	\vspace{0.5\baselineskip} % Whitespace before the editors

	{\scshape\Large Quentin Bergé \\ Marc Ferrière} % Editor list

	\vspace{0.5\baselineskip} % Whitespace below the editor list

	\textit{ENSEEIHT} % Editor affiliation

	\vfill % Whitespace between editor names and publisher logo

	%------------------------------------------------
	%	Publisher
	%------------------------------------------------

	\includegraphics[scale=0.3]{logoN7.png} % changer logo

	\vspace{0.3\baselineskip} % Whitespace under the publisher logo

Fevrier 2019 % Publication year
\end{titlepage}
\newpage

%Table des Matières
\tableofcontents
\newpage

%*******************************************************
% 						DEBUT RAPPORT
%********************************************************


\section{Introduction}
Il est question ici de créer un programme pour effectuer le calcul de trajectoire d'un objet en chute libre et propulsé selon l'enoncé proposé.
Dans ce rapport nous aborderons tout d'abord la modélisation du problème, puis nous nous focaliserons sur l'architecture du programme pour enfin aborder la validation.

\section{Résumé}
\subsection{Résumé en Anglais}
\subsection{Résumé en Français}

\section{Modélisation}

\subsection{Chute Libre}

Si on effectue le bilan des forces sur un objet de masse $m$ possédant une vitesse initiale $v_0$ formant un angle $\alpha$ avec l'axe des abscisses à une altitude initiale $h$ uniquement soumis à la gravité :

\begin{equation*}
\Sigma \vec{F} = m \times \vec{a}
\end{equation*}

Ici $\vec{F}$ se limitera à $\vec{P} = m \vec{g}$ d'où $g=a$

\subsubsection{Solutions Analytiques}

On peut alors obtenir les équations analytiques avec $x(t)$ et $z(t)$ qui nous serviront de références :

\begin{align*}
x(t) &= v_0\cos(\alpha)t \\
z(t) &= v_0\sin(\alpha)t - \frac{gt^2}{2} + h\\
\end{align*}

On peut aussi obtenir les équations qui nous donneront la portée et le temps :

\begin{equation*}
portee = \frac{v_0}{g} \cos(\alpha) \sqrt{v_0 \sin(\alpha) +(v_0\sin(\alpha))^2 + 2gh}
\end{equation*}
\begin{equation*}
t_l = \frac{portee}{v_0\cos(\alpha)}	
\end{equation*}

\subsubsection{Modélisation Euler}
Le schéma d'Euler d'ordre 1 s'écrit : 

\begin{equation*}
y_{k+1} = y_{k}+ \Delta t \frac{df}{dt}(t_{k},y_{k})
\end{equation*}
On utilise un tableau de 4 colonnes noté $y$ qui répertorie les positions et les vitesses comme suit, on obtient alors les équations régissant la dynamique de l'objet en chute libre :


%2 Systèmes d'équations avec équivalence et avec accolade
\begin{equation*}
\left\{
 \begin{array}{c c c}
  y(1,i) =  x(i)\\
  y(2,i) = z(i)\\
  y(3,i) = v_{x}(i)\\ 
  y(4,i) = v_{z}(i)\\
  \end{array}
\qquad
\Longleftrightarrow
\left\{
\begin{array}{c c c}
\frac{dx}{dt} = v_{x}\\
\frac{dz}{dt} = v_{z}\\
\frac{dv_{x}}{dt} = 0\\
\frac{dv_{z}}{dt} = -g\\  
\end{array}
\end{equation*}

On peut alors modéliser notre trajectoire de chute libre par un système du type \label{syst_chute_libre}: 

\begin{equation*}
\begin{cases} 
 y(3,i+1) = y(3,i)\\
 y(4,i+1) = y(4,i)  - dt\times g  \\
 y(1,i+1) = y(1,i)  + dt \times y(3,i)\\
 y(2,i+1) = y(2,i)  + dt \times y(4,i)\\
\end{cases}
\end{equation*}

Avec $dt$ le pas de temps utilisé définit comme :

\begin{equation*}
	dt = \frac{nombredepoints}{tempssimulation} 
\end{equation*}


A chaque itération, on recalcule les vitesses $v_x$ et $v_z$, ainsi que les positions en $x$ et en $z$ à partir du système précédent.
  
\subsubsection{Modélisation Runge-Kutta d'ordre 4}
Le schéma de Runge-Kutta d'ordre 4 est construit comme ceci : \label{2}
\[
y_{k+1} = y_k + \frac{\Delta t}{6} ( k_1 + 2 k_2 + 2 k_3 + k4)
\]
Avec  : 
\begin{equation*}
\begin{cases} 
k_1 = \frac{dy_k}{dt}\\
k_2 = \frac{dy_k}{dt} + \frac{\Delta t }{2}\times k_1\\
k_3 = \frac{dy_k}{dt} + \frac{\Delta t }{2}\times k_2\\
k_4 = \frac{dy_k}{dt} + \Delta t \times k_3\\
\end{cases}
\end{equation*}

On utilisera les dérivées de la modélisation chute libre de la méthode d'Euler \ref{syst_chute_libre} en appliquant la méthode Runge-Kutta à l'ordre 4.


\subsection{Objet Portant Propulsé}

On étudiera ici notre objet de masse $m$ possédant une vitesse initiale $v_0$ formant un angle $\alpha$ avec l'axe des abscisses à une altitude initiale $h$, sera soumis à une force de propulsion $F_0$, une force de portance $L$ et une force de trainée $D$.

\subsubsection{Méthode Euler}
On reprend le même schéma d'Euler à l'ordre 1 (cf. \ref{syst_chute_libre}) en prenant en compte les forces de portance $L$, de traînée $D$ et de propulsion $F_0$ :
\label{3}

\begin{equation*}
\begin{cases} 
\frac{dx}{dt} = v_{x}\\
\frac{dz}{dt} = v_{z}\\
\frac{dv_{x}}{dt} = \frac{  F_0 \cos(\theta) -L\sin(\theta) - D \cos(\theta) }{m}\\
\frac{dv_{z}}{dt} = \frac{ -g + F_0 \sin(\theta) +L\cos(\theta) - D \sin(\theta) }{m}\\
\end{cases}
\end{equation*}

 Avec $\theta $ l'angle entre entre les vecteurs vitesses  $v_x$ et $v_z$ : 
 \begin{equation*}
 	 \theta = \arctan(\frac{v_z}{v_x})
 \end{equation*}

 \subsubsection{Méthode Runge-Kutta d'ordre 4}
 
 On reprend le schéma Runge-Kutta d'ordre 4 (cf \ref{2}) en modélisant les nouvelles dérivées vues en \ref{3}.
 
%******************** 
\section{Architecture}
%********************

Nous détaillerons ici le contenu de l'archive ainsi que l'algorithme du programme.

\subsection{Contenu}

Le programme est placé dans le dossier \verb?Fortran? qui comprend :

\begin{itemize}
	\item un fichier \verb?mod_balistique.f90? % pour ne pas tenir compte de la mise en forme du texte très utile !!!
	\item un fichier de subroutines \verb?subroutines.f90?
	\item le programme \verb?main.f90?
	\item le fichier \verb?Makefile? permettant la compilation\\
	\end{itemize}

Le dossier \verb?RUN? comprend :
\begin{itemize}
	\item les fichier de sorties de forme \verb?BE_Methode_Modele_npt_XXXX_.out? et \verb? Paramétrisation_Alpha_Methode_Modele.out?
	\item le fichier d'entrée \verb?balistique.in?
	\item l'exécutable \verb?main.bin?\\
\end{itemize}

Dans le dossier \verb?Validation? sont compris :
\begin{itemize}
	\item les résultats du programme pour les différents modèles avec les deux modélisation
	\item les graphiques de résultats\\
\end{itemize}

Le dossier \verb?Rapport? comprend les ressources pour le rapport \LaTeX ainsi que le rapport en lui-même.

\subsection{Algorithmie}

Voici l'algorigramme général présentant le fonctionnement du programme.

\begin{figure}[h!]
\bigcenter
\centering
\includegraphics[scale=0.7] {algorigramme}
\caption{Algorigramme général du programme}	
\end{figure}

\subsubsection{Procédure RK4}
Détaillons la procédure pour Runge-Kutta d'ordre 4.\\

On a créé une subroutine nommée \verb?rk4? qui a pour paramètres d'entrées : 
\begin{itemize}
	\item \verb?derivee? : dérivée de la fonction 
	\item \verb?n? : numéro du vecteur que l'on veut calculer
\end{itemize}
Elle calcule la valeur du vecteur $y_{k+1}$ selon le schéma vu en \ref{2}. 
\\

On appelle cette subroutine dans les 2 cas pour calculer les positions et les vitesses en $x$ et en $z$ : 
\begin{itemize}
\item Chute Libre
\item Propulsé	
\end{itemize}

\subsubsection{Paramétrisation d'Angle}
Pour la variation de l'angle $\alpha$, Dans la subroutine \verb?parametrisation_alpha?, on modifie la variable globale de l'angle $\alpha$ de 25 \degre jusqu'à 75 \degre par pas de 5 \degre.
On appelera alors les différentes subroutines présentes dans dans \verb?main.f90? en reprenant l'architecture du programme pour calculer les trajectoires de chaque angle $\alpha$.
On affiche ensuite la portée et le temps de chute pour tous les angles. 


%********************
\section{Validation}
%********************

On effectue la validation selon la solution analytique pour le cas de la chute libre pour les deux méthodes.
Les graphiques sont réalisés dans le \LaTeX à l'aide de gnuplot.

\subsection{Trajectoires}

\subsubsection{Modèle Chute Libre}
Tout d'abord selon la méthode d'Euler.

\begin{figure}[h]
\centering
\begin{gnuplot}[terminal=latex]
set size 1,1; set title "Validation Chute Libre Euler";set xrange [0:180]; set yrange [0:140];set ylabel "z (m)" ; set xlabel "x (m)"; plot "BE_Euler_Chute_Libre_npt_10000.out" using 3:4 title "Méthode Euler" with lines, "BE_Euler_Chute_Libre_npt_10000.out" using 5:6 title "Méthode Analytique" with lines
\end{gnuplot}
\caption{Validation de la trajectoire en Chute Libre selon la méthode d'Euler avec la méthode Analytique}
\end{figure}

On observe la parfaite superposition entre les deux courbes, ainsi on peut valider nos résultats avec la résolution de la méthode d'Euler.

\begin{figure}[h!]
\centering
\begin{gnuplot}[terminal=latex]
set size 1,1; set title "Validation Chute Libre RK4";set xrange [0:180]; set yrange [0:140];set ylabel "z (m)" ; set xlabel "x (m)" ; plot "BE_RK4_Chute_Libre_npt_10000.out" using 3:4 title "Méthode RK4" with lines, "BE_RK4_Chute_Libre_npt_10000.out" using 5:6 title "Méthode Analytique" with lines
\end{gnuplot}
\caption{Validation de la trajectoire en Chute Libre selon la méthode RK4 avec la méthode Analytique}
\end{figure}

\subsubsection{Modèle Propulsé}

\begin{figure}[h!]
\centering
\begin{gnuplot}[terminal=latex]
set size 1,1; set title "Validation Propulsé RK4 Euler";set xrange [0:370]; set yrange [0:140];set ylabel "z (m)" ; set xlabel "x (m)"; plot "BE_RK4_Propulsé_npt_10000.out" using 3:4 title "Méthode RK4" with lines, "BE_Euler_Propulsé_npt_10000.out" using 3:4 title "Méthode Euler" with lines
\end{gnuplot}
\caption{Validation du modèle Propulsé selon la méthode RK4 et la méthode Euler}
\end{figure}


\subsection{Paramétrisation d'Angle}
\subsubsection{Modèle Chute Libre}

\begin{figure}[h!]
\centering
%on utilise un offset puisque que le rotate ne marche pas dans cette configuration
\begin{gnuplot}[terminal=latex]
set size 1,1; set title "Portée en fonction de l'Angle Initial";set xrange [20:80]; set yrange [60:180];set ylabel "portée (m)" offset 2,9; set xlabel "$\alpha$ (degrés)"; plot "Paramétrisation_Alpha_Euler_Chute_Libre.out" using 2:3 title "Méthode Analytique" with lines, "Paramétrisation_Alpha_Euler_Chute_Libre.out" using 2:5 title "Méthode Euler" with lines
\end{gnuplot}
\caption{Validation paramétrisation d'alpha avec la portée en Chute Libre selon la méthode Euler avec la méthode Analytique}
\end{figure}

\begin{figure}[h!]
\centering
\begin{gnuplot}[terminal=latex]
set size 1,1; set title "Portée en fonction de l'Angle Initial";set xrange [20:80]; set yrange [60:180];set ylabel "portée (m)" offset 2,9 ; set xlabel "$\alpha$ (degrés)"; plot "Paramétrisation_Alpha_Euler_Chute_Libre.out" using 2:3 title "Méthode Analytique" with lines, "Paramétrisation_Alpha_RK4_Chute_Libre.out" using 2:5 title "Méthode RK4" with lines
\end{gnuplot}
\caption{Validation paramétrisation d'alpha avec la portée en Chute Libre selon la méthode RK4 avec la méthode Analytique}
\end{figure}

\begin{figure}[h!]
\centering
\begin{gnuplot}[terminal=latex]
set size 1,1; set title "Temps en fonction de l'Angle Initial";set xrange [20:80]; set yrange [4:10];set ylabel "t(s)" ; set xlabel "$\alpha$ (degrés)"; plot "Paramétrisation_Alpha_Euler_Chute_Libre.out" using 2:4 title "Méthode Analytique" with lines, "Paramétrisation_Alpha_Euler_Chute_Libre.out" using 2:6 title "Méthode Euler" with lines
\end{gnuplot}
\caption{Validation paramétrisation d'alpha avec le temps en Chute Libre selon la méthode Euler avec la méthode Analytique}
\end{figure}


\begin{figure}[h!]
\centering
\begin{gnuplot}[terminal=latex]
set size 1,1; set title "Temps en fonction l'Angle Initial";set xrange [20:80]; set yrange [4:10];set ylabel "t(s)" ; set xlabel "$\alpha$ (degrés)"; plot "Paramétrisation_Alpha_Euler_Chute_Libre.out" using 2:4 title "Méthode Analytique" with lines, "Paramétrisation_Alpha_RK4_Chute_Libre.out" using 2:6 title "Méthode RK4" with lines
\end{gnuplot}
\caption{Validation paramétrisation d'alpha avec le temps en Chute Libre selon la méthode RK4 avec la méthode Analytique}
\end{figure}

\subsubsection{Modèle Propulsé}
Passons maintenant au modèle Propulsé.

\begin{figure}[h!]
\centering

\begin{gnuplot}[terminal=latex]
set size 1,1; set title "Portée en fonction de l'Angle Initial";set xrange [20:80]; set yrange [250:450];set ylabel "portée (m)" offset 2,9 ; set xlabel "$\alpha$ (degrés)"; plot "Paramétrisation_Alpha_Euler_Propulsé.out" using 2:3 title "Méthode Euler" with lines, "Paramétrisation_Alpha_RK4_Propulsé.out" using 2:3 title "Méthode RK4" with lines
\end{gnuplot}
\caption{Validation paramétrisation d'alpha avec la portée en Propulsé selon la méthode Euler avec la méthode Analytique}
\end{figure}




\begin{figure}[h!]
\centering
\begin{gnuplot}[terminal=latex]
set size 1,1; set title "Temps en fonction l'Angle Initial";set xrange [20:80]; set yrange [4:10];set ylabel "t(s)" ; set xlabel "$\alpha$ (degrés)"; plot "Paramétrisation_Alpha_Euler_Propulsé.out" using 2:4 title "Méthode Euler" with lines, "Paramétrisation_Alpha_RK4_Propulsé.out" using 2:4 title "Méthode RK4" with lines
\end{gnuplot}
\caption{Validation paramétrisation d'alpha avec le temps en Propulsé selon la méthode RK4 avec la méthode Analytique}
\end{figure}

\newpage

\section{Conclusion}


\end{document}
